\documentclass[11pt, dvipsnames, svgnames, x11names]{article}

% URLs and hyperlinks ---------------------------------------
\usepackage{hyperref}
\hypersetup{
    colorlinks=true,
    linkcolor=blue,
    filecolor=magenta,      
    urlcolor=black,
}
\usepackage{xurl}
%---------------------------------------------------

% footnotes in headings -------------------------------------
\usepackage[stable]{footmisc}
%----------------------------------------------------

\usepackage{float}
\usepackage{listings}
\usepackage{color}
\usepackage{xcolor}

\definecolor{dkgreen}{rgb}{0,0.6,0}
\definecolor{gray}{rgb}{0.5,0.5,0.5}
\definecolor{mauve}{rgb}{0.58,0,0.82}

\lstset{frame=tb,
    language=vhdl,
    aboveskip=3mm,
    belowskip=3mm,
    showstringspaces=false,
    columns=flexible,
    basicstyle=\ttfamily,
    numbers=left,
    numberstyle=\small\color{gray},
    keywordstyle=\bfseries\color{Green4},
    commentstyle=\color{gray},
    stringstyle=\color{mauve},
    breaklines=true,
    breakatwhitespace=true,
    tabsize=4,
    identifierstyle=\color{black}
}

\usepackage{xepersian}
\settextfont{Yas}
\setdigitfont{Yas}

\newcommand{\mahdi}{مهدی حق‌وردی }
\newcommand{\zohre}{زهره سورانی}

\title{تمرین فصل دوم - دستورات: زبان کامپیوتر}
\author{
    زهره سورانی \\
    مهدی‌ حق‌وردی}
\date{}

\begin{document}
\maketitle    
\begin{abstract}        
سوالات فصل اول کتاب، که آموزش \lr{ISA}ی \lr{MIPS} است، برای شما تالیف شده‌اند.

پاسخ هر سوال را در قسمت مربوط آنها در کوئرا به صورت \lr{PDF} به صورت تایپ‌ شده، یا دست‌نویس خوش‌خط و خوانا آپلود کنید.               

هر سوال دارای یک پاورقی‌ست که طراح آن سوال را مشخص می‌کند، برای پرسیدن سوالات‌ خود به طراح هر سوال مراجعه کنید.               

پس از پایان‌ یافتن زمان ارسال تمرین، پاسخ‌های این تمرین در آدرس زیر قرار خواهد گرفت.
\begin{flushleft}
\url{https://github.com/mahdihaghverdi/arch-questions-answers/blob/main/instructions-language-of-the-computer}
\end{flushleft}

\end{abstract}
\tableofcontents
\newpage

\section{\lr{C} به \lr{Assembly}}
تابع زیر را به زبان اسمبلی بنویسید.
\begin{latin}
\begin{lstlisting}[language=c]
void copy(int a[], int b[], int n){
    int i;
    for(i=0; i!=n; i++) {
        a[i]=b[i];
    }
}
\end{lstlisting}
\end{latin}
\section{عملیات‌های منطقی}
سوالات زیر را بر اساس جدول زیر که محتویات رجیستر‌های \lr{\texttt{\$t0}}
و 
\lr{\texttt{\$t1}}
را داراست، بنویسید.

\begin{latin}
\begin{table}[H]
\begin{center}
\begin{tabular}{|c|c|}
\hline
\texttt{\$t0 = 0xAAAAAAAA} & \texttt{\$t1 = 0x12345678} \\
\hline
\end{tabular}
\end{center}
\end{table}
\end{latin}

\begin{enumerate}
\item 
مقدار \lr{\texttt{\$t2}} بعد از اجرای کد زیر چقدر است؟ پاسخ را به صورت \lr{hexadeciaml} بنویسید.

\begin{latin}
\begin{lstlisting}
sll $t2, $t0, 44
or $t2, $t2, $t1
\end{lstlisting}
\end{latin}
\item 
مقدار \lr{\texttt{\$t2}} بعد از اجرای کد زیر چقدر است؟ پاسخ را به صورت \lr{hexadeciaml} بنویسید.

\begin{latin}
\begin{lstlisting}[keywords={sll, andi}]
sll $t2, $t0, 4
andi $t2, $t2, –1
\end{lstlisting}
\end{latin}

\item 
مقدار \lr{\texttt{\$t2}} بعد از اجرای کد زیر چقدر است؟ پاسخ را به صورت \lr{hexadeciaml} بنویسید.

\begin{latin}
\begin{lstlisting}[keywords={srl, andi}]
srl $t2, $t0, 3
andi $t2, $t2, 0xFFEF
\end{lstlisting}
\end{latin}
\end{enumerate}

\section{عملیات‌های شرطی \lr{Branches}}
\subsection{مقدار نهایی چیست؟}
کد‌هایی زیر را در نظر بگیرید
\begin{latin}
\begin{enumerate}
\item
\begin{lstlisting}[keywords={addi, subi, bne}]
LOOP:   addi $s2, $s2, 2
        subi $t1, $t1, 1
        bne $t1, $0, LOOP
DONE:
\end{lstlisting}

\item
\begin{lstlisting}[keywords={slt, beq, subi, addi, j}]
LOOP:   slt $t2, $0, $t1
        beq $t2, $0, DONE
        subi $t1, $t1, 1
        addi $s2, $s2, 2
        j LOOP
DONE:
\end{lstlisting}
\end{enumerate}
\end{latin}

فرض کنید که مقدار اولیه‌ی رجیستر
\lr{\texttt{\$t1}}
برابر است با 
\lr{\texttt{10}}
و مقدار اولیه رجیستر \lr{\texttt{\$s2}} برابر با صفر، مقدار نهایی 
\lr{\texttt{\$s2}}
چند است؟
\subsection{کد اسمبلی را بنویسید.}
کد اسمبلی کد زیر را بنویسید.
\begin{latin}
\begin{lstlisting}[language=c]
for(i=0; i<a; i++)
    for(j=0; j<b; j++)
        D[4*j] = i + j;
\end{lstlisting}
\end{latin}

\section{توابع}
\subsection{تابع \lr{\texttt{positive}}}
\subsection{\lr{In-line} کردن تابع}
\subsection{چندین‌ بار صدا زده شدن تابع}

\section{توابع بازگشتی}
\subsection{تابع فاکتوریل}
\subsection{تابع فیبوناچی}

\end{document}
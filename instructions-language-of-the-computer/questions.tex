\documentclass[11pt, dvipsnames, svgnames, x11names]{article}

% URLs and hyperlinks ---------------------------------------
\usepackage{hyperref}
\hypersetup{
    colorlinks=true,
    linkcolor=blue,
    filecolor=magenta,      
    urlcolor=black,
}
\usepackage{xurl}
%---------------------------------------------------

% footnotes in headings -------------------------------------
\usepackage[stable]{footmisc}
%----------------------------------------------------

\usepackage{float}
\usepackage{listings}
\usepackage{color}
\usepackage{xcolor}

\definecolor{dkgreen}{rgb}{0,0.6,0}
\definecolor{gray}{rgb}{0.5,0.5,0.5}
\definecolor{mauve}{rgb}{0.58,0,0.82}

\lstset{frame=tb,
    language=vhdl,
    aboveskip=3mm,
    belowskip=3mm,
    showstringspaces=false,
    columns=flexible,
    basicstyle=\ttfamily,
    numbers=left,
    numberstyle=\small\color{gray},
    keywordstyle=\bfseries\color{Green4},
    commentstyle=\color{gray},
    stringstyle=\color{mauve},
    breaklines=true,
    breakatwhitespace=true,
    tabsize=4,
    identifierstyle=\color{black}
}

\usepackage{xepersian}
\settextfont{Yas}
\setdigitfont{Yas}

\newcommand{\mahdi}{مهدی حق‌وردی }
\newcommand{\zohre}{زهره سورانی}

\title{تمرین فصل دوم - دستورات: زبان کامپیوتر}
\author{
    زهره سورانی \\
    مهدی‌ حق‌وردی}
\date{}

\begin{document}
\maketitle    
\begin{abstract}        
سوالات فصل دوم کتاب، که آموزش \lr{ISA}ی \lr{MIPS} است، برای شما تالیف شده‌اند.

پاسخ هر سوال را در قسمت مربوط آنها در کوئرا به صورت \lr{PDF} به صورت تایپ‌ شده، یا دست‌نویس خوش‌خط و خوانا آپلود کنید.               

هر سوال دارای یک پاورقی‌ست که طراح آن سوال را مشخص می‌کند، برای پرسیدن سوالات‌ خود به طراح هر سوال مراجعه کنید.               

پس از پایان‌ یافتن زمان ارسال تمرین، پاسخ‌های این تمرین در آدرس زیر قرار خواهد گرفت.
\begin{flushleft}
\url{https://github.com/mahdihaghverdi/arch-questions-answers/tree/main/instructions-language-of-the-computer}
\end{flushleft}

\end{abstract}
\tableofcontents
\newpage

\section[\lr{C} به \lr{Assembly}]{\lr{C} به \lr{Assembly}\RTLfootnote{\mahdi}}
تابع زیر را به زبان اسمبلی بنویسید.
\begin{latin}
\begin{lstlisting}[language=c]
void copy(int a[], int b[], int n){
    int i;
    for(i=0; i!=n; i++) {
        a[i]=b[i];
    }
}
\end{lstlisting}
\end{latin}
\section[عملیات‌های منطقی]{عملیات‌های منطقی\RTLfootnote{\zohre}}
سوالات زیر را بر اساس جدول زیر که محتویات رجیستر‌های \lr{\texttt{\$t0}}
و 
\lr{\texttt{\$t1}}
را داراست، بنویسید.

\begin{latin}
\begin{table}[H]
\begin{center}
\begin{tabular}{|c|c|}
\hline
\texttt{\$t0 = 0xAAAAAAAA} & \texttt{\$t1 = 0x12345678} \\
\hline
\end{tabular}
\end{center}
\end{table}
\end{latin}

\begin{enumerate}
\item 
مقدار \lr{\texttt{\$t2}} بعد از اجرای کد زیر چقدر است؟ پاسخ را به صورت \lr{hexadeciaml} بنویسید.

\begin{latin}
\begin{lstlisting}
sll   $t2, $t0, 44
or    $t2, $t2, $t1
\end{lstlisting}
\end{latin}
\item 
مقدار \lr{\texttt{\$t2}} بعد از اجرای کد زیر چقدر است؟ پاسخ را به صورت \lr{hexadeciaml} بنویسید.

\begin{latin}
\begin{lstlisting}[keywords={sll, andi}]
sll   $t2, $t0, 4
andi  $t2, $t2, –1
\end{lstlisting}
\end{latin}

\item 
مقدار \lr{\texttt{\$t2}} بعد از اجرای کد زیر چقدر است؟ پاسخ را به صورت \lr{hexadeciaml} بنویسید.

\begin{latin}
\begin{lstlisting}[keywords={srl, andi}]
srl   $t2, $t0, 3
andi  $t2, $t2, 0xFFEF
\end{lstlisting}
\end{latin}
\end{enumerate}

\section{عملیات‌های شرطی \lr{(Branches)}}
\subsection[مقدار نهایی چیست؟]{مقدار نهایی چیست؟\RTLfootnote{\zohre}}
کد‌هایی زیر را در نظر بگیرید
\begin{latin}
\begin{enumerate}
\item
\begin{lstlisting}[keywords={addi, subi, bne}]
LOOP:   addi   $s2, $s2, 2
        subi   $t1, $t1, 1
        bne    $t1, $0, LOOP
DONE:
\end{lstlisting}

\item
\begin{lstlisting}[keywords={slt, beq, subi, addi, j}]
LOOP:   slt    $t2, $0, $t1
        beq    $t2, $0, DONE
        subi   $t1, $t1, 1
        addi   $s2, $s2, 2
        j      LOOP
DONE:
\end{lstlisting}
\end{enumerate}
\end{latin}

فرض کنید که مقدار اولیه‌ی رجیستر
\lr{\texttt{\$t1}}
برابر است با 
\lr{\texttt{10}}
و مقدار اولیه رجیستر \lr{\texttt{\$s2}} برابر با صفر، مقدار نهایی 
\lr{\texttt{\$s2}}
چند است؟
\subsection[کد اسمبلی را بنویسید.]{کد اسمبلی را بنویسید.\RTLfootnote{\mahdi}}
کد اسمبلی کد زیر را بنویسید.
\begin{latin}
\begin{lstlisting}[language=c]
for(i=0; i<a; i++)
    for(j=0; j<b; j++)
        D[4*j] = i + j;
\end{lstlisting}
\end{latin}

\section{توابع}
\subsection[تابع \lr{\texttt{positive}}]{تابع \lr{\texttt{positive}}\RTLfootnote{\zohre}}
کد اسمبلی تابع زیر را بنویسید.
\begin{latin}
\begin{lstlisting}[language=c]
int positive(int a, int b) {
    if (addit(a, b) > 0)
       return 1;
    else
        return 0;
}
int addit(int a, int b) {return a+b;}
\end{lstlisting}
\end{latin}

\subsection[\lr{In-line} کردن تابع]{\lr{In-line} کردن تابع\RTLfootnote{\zohre}}
کامپایلر‌های زبان‌های برنامه‌نویسی می‌توانند توابعی را \lr{in-line} کنند. این به این معناست که کد تابع را در دل جایی که نوشته شده‌اند، تزریق میکنند و از یک 
\lr{function call}
جلوگیری می‌کنند.
حال شما کد تابع \lr{addit} را 
\lr{in-line}
کنید.
\subsection[چندین‌ بار صدا زده شدن تابع]{چندین‌ بار صدا زده شدن تابع\RTLfootnote{\mahdi}}
کد زیر را به اسمبلی بنویسید. (نیازی به دانستن بدنه‌ی تابع \lr{func} نیست، صرفا آن را به صورت یک \lr{label} تعریف کنید)
\begin{latin}
\begin{lstlisting}[language=c]
int f(int a, int b, int c, int d){
    return func(func(a,b), c+d);
}
\end{lstlisting}
\end{latin}

\section{توابع بازگشتی}
\subsection[تابع فاکتوریل]{تابع فاکتوریل\RTLfootnote{\zohre}}
کد اسمبلی زیر، یک کد \underline{دارای اشتباه} از پیاده‌سازی فاکتوریل است، آنرا اصلاح کنید. مقدار اولیه پاس داده شده به تابع، در رجیستر
\lr{\texttt{\$a0}}
ذخیره شده و نتیجه در رجیستر
\lr{\texttt{\$v0}}
ذخیره می‌شود.

\begin{latin}
\begin{lstlisting}[keywords={sw, addi, slti, beq, jr, jal, lw, mul}]
FACT:   sw      $ra, 4($sp)
        sw      $a0, 0($sp)
        addi    $sp, $sp, -8
        slti    $t0, $a0, 1
        beq     $t0, $0, L1
        addi    $v0, $0, 1
        addi    $sp, $sp, 8
        jr      $ra
        
L1:     addi    $a0, $a0, -1
        jal     FACT
        addi    $sp, $sp, 8
        lw      $a0, 0($sp)
        lw      $ra, 4($sp)
        mul     $v0, $a0, $v0
        jr      $ra
\end{lstlisting}
\end{latin}
\subsection[تابع فیبوناچی]{تابع فیبوناچی\RTLfootnote{\mahdi}}
کد اسمبلی زیر، یک کد \underline{دارای اشتباه} از پیاده‌سازی دنباله‌ی فیبوناچی است، آنرا اصلاح کنید. مقدار اولیه پاس داده شده به تابع، در رجیستر
\lr{\texttt{\$a0}}
ذخیره شده و نتیجه در رجیستر
\lr{\texttt{\$v0}}
ذخیره می‌شود.

\begin{latin}
\begin{lstlisting}[keywords={addi, sw, slti, beq, j, jal, lw, jr}]
FIB:    addi    $sp, $sp, –12
        sw      $ra, 0($sp)
        sw      $s1, 4($sp)
        sw      $a0, 8($sp)
        slti    $t0, $a0, 1
        beq     $t0, $0, L1
        addi    $v0, $a0, $0
        j       EXIT
        
L1:     addi    $a0, $a0, –1
        jal     FIB
        addi    $s1, $v0, $0
        addi    $a0, $a0, –1
        jal     FIB
        add     $v0, $v0, $s1
        
EXIT:   lw      $ra, 0($sp)
        lw      $a0, 8($sp)
        lw      $s1, 4($sp)
        addi    $sp, $sp, 12
        jr      $ra
\end{lstlisting}
\end{latin}
\end{document}
\documentclass[11pt, dvipsnames, svgnames, x11names]{article}

% URLs and hyperlinks ---------------------------------------
\usepackage{hyperref}
\hypersetup{
    colorlinks=true,
    linkcolor=blue,
    filecolor=magenta,      
    urlcolor=black,
}
\usepackage{xurl}
%---------------------------------------------------

% footnotes in headings -------------------------------------
\usepackage[stable]{footmisc}
%----------------------------------------------------

\usepackage{float}
\usepackage{listings}
\usepackage{color}
\usepackage{xcolor}
\usepackage{adjustbox}
\usepackage{makecell}

\definecolor{dkgreen}{rgb}{0,0.6,0}
\definecolor{gray}{rgb}{0.5,0.5,0.5}
\definecolor{mauve}{rgb}{0.58,0,0.82}

\lstset{frame=tb,
    language=vhdl,
    aboveskip=3mm,
    belowskip=3mm,
    showstringspaces=false,
    columns=flexible,
    basicstyle=\ttfamily,
    numbers=left,
    numberstyle=\small\color{gray},
    keywordstyle=\bfseries\color{Green4},
    commentstyle=\color{gray},
    stringstyle=\color{mauve},
    breaklines=true,
    breakatwhitespace=true,
    tabsize=4,
    identifierstyle=\color{black}
}

\usepackage{xepersian}
\settextfont{Yas}
\setdigitfont{Yas}

\title{پاسخ تمرین سری سوم}
\date{}

\begin{document}
\maketitle
\tableofcontents
\newpage
\section{سیگنال‌های کنترلی در یک پردازنده‌ی \lr{Single Cycle}}
\begin{latin}
\begin{table}[H]
\begin{adjustbox}{width=\textwidth}
\begin{tabular}{|c|c|c|c|c|c|c|c|}
\hline
&
RegWrite &
MemRead&
ALUMux&
MemWrite&
ALUOp&
RegMux&
Branch\\
\hline
\hline
a. &
1 &
0 &
0 (Reg)&
0 &
AND &
1 (ALU)&
0 \\
\hline
b. &
0 &
0 &
1 (Imm)&
1 &
ADD &
X &
0 \\
\hline
\end{tabular}
\end{adjustbox}
\end{table}
\end{latin}

\section{پیش‌نیاز‌‌های داده‌ای}
\subsection{وابستگی‌ها}
\begin{latin}
\begin{table}[H]
\begin{adjustbox}{width=\textwidth}
\begin{tabular}{|c|l|l|}
\hline
& Instruction Sequence & Dependencies \\
\hline
a. &
\makecell[l]{\texttt{I1: SW R16, –100(R6)} \\ \texttt{I2: LW  R4, 8(R16)} \\ \texttt{I3: ADD R5, R4, R4}}
& \makecell[l]{\texttt{\texttt{RAW}} on \texttt{R4} from \texttt{I2} to \texttt{I3}} \\
\hline
b. &
\makecell[l]{\texttt{I1: OR R1, R2, R3} \\ \texttt{I2: OR R2, R1, R4} \\ \texttt{I3: OR R1, R1, R2}} &
\makecell[l]{\texttt{\texttt{RAW}} on \texttt{R1} from \texttt{I1} to \texttt{I2} and \texttt{I3} \\ \texttt{RAW} on \texttt{R2} from \texttt{I2} to \texttt{I3} \\ \texttt{WAR} on \texttt{R2} from \texttt{I1} to \texttt{I2} \\ \texttt{WAR} on \texttt{R1} from \texttt{I2} to \texttt{I3} \\ \texttt{WAW} on \texttt{R1} from \texttt{I1} to \texttt{I3}}
\\
\hline
\end{tabular}
\end{adjustbox}
\end{table}
\end{latin}
برای مطالعه‌ی معنی وابستگی‌ها به این لینک مراجعه کنید:
\begin{flushleft}
\url{https://en.wikipedia.org/wiki/Data_dependency}
\end{flushleft}

\subsection{افزودن \lr{NOOP} بدون وجود \lr{Full Forwarding}}
\begin{latin}
\begin{table}[H]
\begin{center}
\begin{tabular}{|c|l|l|}
\hline
& Instruction Sequence & \\
\hline
a. &
\makecell[l]{
\texttt{SW R16, –100(R6)} \\
\texttt{LW  R4, 8(R16)} \\
\texttt{NOOP} \\
\texttt{NOOP} \\
\texttt{ADD R5, R4, R4}}&
Delay \texttt{I3} to avoid \texttt{RAW} hazard on \texttt{R4} from \texttt{I2}\\
\hline
b. &
\makecell[l]{
\texttt{OR R1, R2, R3} \\
\texttt{NOOP} \\
\texttt{NOOP} \\
\texttt{OR R2, R1, R4} \\
\texttt{NOOP} \\
\texttt{NOOP} \\
\texttt{OR R1, R1, R2} \\
} &
\makecell[l]{
Delay \texttt{I2} to avoid \texttt{RAW} hazard on \texttt{R1} from \texttt{I1} \\ \\ \\
Delay \texttt{I3} to avoid \texttt{RAW} hazard on \texttt{R2} from \texttt{I2}}
\\
\hline
\end{tabular}
\end{center}
\end{table}
\end{latin}

\subsection{افزون \lr{NOOP} با وجود \lr{Full Forwarding}}
\section{نوشتن یک \lr{Pipeline Stage}}
\section{رجیستر‌های \lr{Pipeline}ها}
\subsection{رجیستر‌های \lr{Pipeline}}
\subsection{اتفاقات هنگام اجرا}
\section{\lr{Execution} در یک \lr{CPU} عه \lr{Pipeline} شده}
\section{\lr{Hazard}ها}
\section{ارتباط بین \lr{Forwarding}، \lr{Hazard} و \lr{ISA Desing}}
\end{document}
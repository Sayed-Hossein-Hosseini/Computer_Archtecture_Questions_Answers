\documentclass[11pt, dvipsnames, svgnames, x11names]{article}

% URLs and hyperlinks ---------------------------------------
\usepackage{hyperref}
\hypersetup{
    colorlinks=true,
    linkcolor=blue,
    filecolor=magenta,      
    urlcolor=black,
}
\usepackage{xurl}
%----------------------------------------------------

% footnotes in headings -----------------------------
\usepackage[stable]{footmisc}
%----------------------------------------------------

\usepackage{float}
\usepackage{listings}
\usepackage{color}
\usepackage{xcolor}

\definecolor{dkgreen}{rgb}{0,0.6,0}
\definecolor{gray}{rgb}{0.5,0.5,0.5}
\definecolor{mauve}{rgb}{0.58,0,0.82}

\lstset{frame=tb,
    language=vhdl,
    aboveskip=3mm,
    belowskip=3mm,
    showstringspaces=false,
    columns=flexible,
    basicstyle=\ttfamily,
    numbers=left,
    numberstyle=\small\color{gray},
    keywordstyle=\bfseries\color{Green4},
    commentstyle=\color{gray},
    stringstyle=\color{mauve},
    breaklines=true,
    breakatwhitespace=true,
    tabsize=4,
    identifierstyle=\color{black}
}

\usepackage{xepersian}
\settextfont{Yas}
\setdigitfont{Yas}

\newcommand{\mahdi}{مهدی حق‌وردی }
\newcommand{\zohre}{زهره سورانی}

\title{تمرین فصل سوم - پردازنده}
\author{
حسنا رجایی \\
    سجاد شیروانی \\
    مهدی‌ حق‌وردی \\
سید حسین حسینی}
\date{}

\begin{document}
\maketitle    
\begin{abstract}        
سوالات فصل سوم کتاب، که در مورد خود پردازنده صحبت می‌کند، برای شما آماده شده‌اند.

پاسخ هر سوال را در قسمت مربوط آنها در کوئرا به صورت \lr{PDF} به صورت تایپ‌ شده، یا دست‌نویس خوش‌خط و خوانا آپلود کنید.               

پس از پایان‌ یافتن زمان ارسال تمرین، پاسخ‌های این تمرین در آدرس زیر قرار خواهد گرفت.
\begin{flushleft}
\url{https://github.com/mahdihaghverdi/arch-questions-answers/tree/main/the-processor}
\end{flushleft}

\end{abstract}
\tableofcontents
\newpage

\section{سیگنال‌های کنترلی در یک پردازنده‌ی \lr{Single Cycle}}

\section{پیش‌نیاز‌‌های داده‌ای}
\subsection{}
\subsection{}
\subsection{}

\section{نوشتن یک \lr{Pipeline Stage}}

\section{رجیستر‌های \lr{Pipeline}ها}

\section{\lr{Execution} در یک \lr{CPU} عه \lr{Pipeline} شده}

\section{\lr{Hazard}ها}

\section{ارتباط بین \lr{Forwarding}، \lr{Hazard} و \lr{ISA Desing}}
\end{document}
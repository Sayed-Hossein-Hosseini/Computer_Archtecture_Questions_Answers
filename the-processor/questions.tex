\documentclass[11pt, dvipsnames, svgnames, x11names]{article}

% URLs and hyperlinks ---------------------------------------
\usepackage{hyperref}
\hypersetup{
    colorlinks=true,
    linkcolor=blue,
    filecolor=magenta,      
    urlcolor=black,
}
\usepackage{xurl}
%----------------------------------------------------

% footnotes in headings -----------------------------
\usepackage[stable]{footmisc}
%----------------------------------------------------

\usepackage{float}
\usepackage{listings}
\usepackage{color}
\usepackage{xcolor}
\usepackage{makecell}

\definecolor{dkgreen}{rgb}{0,0.6,0}
\definecolor{gray}{rgb}{0.5,0.5,0.5}
\definecolor{mauve}{rgb}{0.58,0,0.82}

\lstset{frame=tb,
    language=vhdl,
    aboveskip=3mm,
    belowskip=3mm,
    showstringspaces=false,
    columns=flexible,
    basicstyle=\ttfamily,
    numbers=left,
    numberstyle=\small\color{gray},
    keywordstyle=\bfseries\color{Green4},
    commentstyle=\color{gray},
    stringstyle=\color{mauve},
    breaklines=true,
    breakatwhitespace=true,
    tabsize=4,
    identifierstyle=\color{black}
}

\usepackage{xepersian}
\settextfont{Yas}
\setdigitfont{Yas}

\newcommand{\mahdi}{مهدی حق‌وردی }
\newcommand{\zohre}{زهره سورانی}

\title{تمرین فصل سوم - پردازنده}
\author{
حسنا رجایی \\
    سجاد شیروانی \\
    مهدی‌ حق‌وردی \\
سید حسین حسینی}
\date{}

\begin{document}
\maketitle    
\begin{abstract}        
سوالات فصل سوم کتاب، که در مورد خود پردازنده صحبت می‌کند، برای شما آماده شده‌اند.

پاسخ هر سوال را در قسمت مربوط آنها در کوئرا به صورت \lr{PDF} به صورت تایپ‌ شده، یا دست‌نویس خوش‌خط و خوانا آپلود کنید.               

پس از پایان‌ یافتن زمان ارسال تمرین، پاسخ‌های این تمرین در آدرس زیر قرار خواهد گرفت.
\begin{flushleft}
\url{https://github.com/mahdihaghverdi/arch-questions-answers/tree/main/the-processor}
\end{flushleft}

\end{abstract}
\tableofcontents
\newpage

\section{سیگنال‌های کنترلی در یک پردازنده‌ی \lr{Single Cycle}}

سیگنال‌های کنترلی تولید شده در اجرای این دستورات را ذکر کنید.
\begin{latin}
\begin{table}[H]
\begin{center}
\begin{tabular}{|c|c|c|}
\hline
& Instruction & Interpretation \\
\hline
a. &
\texttt{AND Rd, Rs, Rt} &
\texttt{Reg[Rd] = Reg[Rs] AND Reg[Rt]} \\
\hline
b. &
\texttt{SW Rt, Offs(Rs)} &
\texttt{Mem[Reg[Rs] + Offs] = Reg[Rt]} \\
\hline
\end{tabular}
\end{center}
\end{table}
\end{latin}

\section{پیش‌نیاز‌‌های داده‌ای}
دستورات زیر را در نظر بگیرید
\begin{latin}
\begin{table}[H]
\begin{center}
\begin{tabular}{|c|l|}
\hline
& Instruction Sequence \\
\hline
a. &
\makecell[l]{\texttt{SW R16, –100(R6)} \\ \texttt{LW  R4, 8(R16)} \\ \texttt{ADD R5, R4, R4}} \\
\hline
b. &
\makecell[l]{\texttt{OR R1, R2, R3} \\ \texttt{OR R2, R1, R4} \\ \texttt{OR R1, R1, R2}}\\
\hline
\end{tabular}
\end{center}
\end{table}
\end{latin}

\subsection{وابستگی‌ها}
تمامی وابستگی‌ها در این سلسله دستورات را بنویسید.

\subsection{افزودن \lr{NOOP} بدون وجود \lr{Full Forwarding}}
فرض کنید هیچ فورواردینگی در پردازنده وجود ندارد، ابتدا بگویید کجا
\lr{hazard}
رخ می‌دهد و بین دستورات دستور 
\lr{NOOP}
بگذارید تا اجرا به درستی انجام بشود.

\subsection{افزون \lr{NOOP} با وجود \lr{Full Forwarding}}
حالا فرض کنید در این پردازنده، فول فورواردینگ وجود دارد، ابتدا بگویید کجا
\lr{hazard}
رخ می‌دهد و بین دستورات دستور 
\lr{NOOP}
بگذارید تا اجرا به درستی انجام بشود.

\section{نوشتن یک \lr{Pipeline Stage}}
برای دستورات زیر، یک نمودار
\lr{Pipeline Stage}
بکشید، فرض کنید که در این پردازنده‌، \lr{Full forwarding} داریم.

\begin{latin}
\begin{table}[H]
\begin{center}
\begin{tabular}{|c|l|}
\hline
& Instruction Sequence \\
\hline
a. &
\makecell[l]{
\texttt{SW R16, 12(R6)} \\ 
\texttt{LW R16, 8(R6)} \\ 
\texttt{BEQ R5, R4, Label ; Assume R5 != R4} \\
\texttt{ADD R5, R1, R4} \\ 
\texttt{SLT R5, R15, R4}} \\
\hline
b. &
\makecell[l]{
\texttt{SW R2, 0(R3)} \\ 
\texttt{OR R1, R2, R3} \\ 
\texttt{BEQ R2, R0, Label ; Assume R2 == R0} \\
\texttt{OR R2, R2, R0} \\
\texttt{Label: ADD R1, R4, R3}}\\
\hline
\end{tabular}
\end{center}
\end{table}
\end{latin}

\section{رجیستر‌های \lr{Pipeline}ها}
دستورات زیر را در نظر بگیرید
\begin{latin}
\begin{table}[H]
\begin{center}
\begin{tabular}{|c|l|}
\hline
& Instruction Sequence \\
\hline
a. &
\texttt{SW R16, –100(R6)} \\
\hline
b. &
\texttt{OR R2, R1, R0} \\ 
\hline
\end{tabular}
\end{center}
\end{table}
\end{latin}
\subsection{رجیستر‌های \lr{Pipeline}}
وقتی که این دستورات اجرا می‌شوند، که داده‌هایی در ریجستر‌های این پردازنده‌ی پایپ‌لاین شده ذخیره می‌شوند؟

\subsection{اتفاقات هنگام اجرا}
چه اتفاقی حین مرحله‌ی \lr{Execution} و \lr{Memory} می‌افتد؟

\section{\lr{Execution} در یک \lr{CPU} عه \lr{Pipeline} شده}

\section{\lr{Hazard}ها}

\section{ارتباط بین \lr{Forwarding}، \lr{Hazard} و \lr{ISA Desing}}
\end{document}
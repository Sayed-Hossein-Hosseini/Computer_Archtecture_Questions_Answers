\documentclass[11pt]{article}

% URLs and hyperlinks ---------------------------------------
\usepackage{hyperref}
\hypersetup{
    colorlinks=true,
    linkcolor=blue,
    filecolor=magenta,      
    urlcolor=black,
}
\usepackage{xurl}
%---------------------------------------------------

% footnotes in headings -------------------------------------
\usepackage[stable]{footmisc}
%------------------------------------------------------------

\usepackage{float}

\usepackage{xepersian}
\settextfont{Yas}
\setdigitfont{Yas}

\newcommand{\mahdi}{مهدی حق‌وردی}
\newcommand{\hosna}{حسنا رجایی}
\newcommand{\isa}{\lr{ISA} }
\newcommand{\clockrate}{\lr{Clock Rate} }
\newcommand{\cpi}{\lr{CPI} }

\title{تمرینات فصل اول - مقدمات و \lr{Performance}}
\author{
    حسنا رجایی \\
مهدی‌ حق‌وردی}
\date{}

\begin{document}
    \maketitle    
    \begin{abstract}
        سوالات فصل اول کتاب، که توضیحات کلی در مورد سخت‌افزار کامپیوتر و مفاهیم مهمی که در این درس با آنها کار داریم است، برای شما تالیف شده‌اند.
        
        پاسخ هر سوال را در قسمت مربوط آنها در کوئرا به صورت \lr{PDF} به صورت تایپ‌ شده، یا دست‌نویس خوش‌خط و خوانا آپلود کنید.
        
        هر سوال دارای یک پاورقی‌ست که طراح آن سوال را مشخص می‌کند، برای پرسیدن سوالات‌ خود به طراح هر سوال مراجعه کنید.
        
        پس از پایان‌ یافتن زمان ارسال تمرین، پاسخ‌های این تمرین در آدرس زیر قرار خواهد گرفت.
        \begin{flushleft}
            \url{https://github.com/mahdihaghverdi/arch-questions-answers/blob/main/computer-abstractions-and-technology/}
        \end{flushleft}

        
    \end{abstract}
    \tableofcontents
    \newpage
    
\section[مقدمات و تعاریف]{مقدمات و تعاریف\RTLfootnote{\hosna}}
    ابتدا مفاهیم 
    \lr{CPI},
    \lr{IC (Instruction Count)}
    و
    \clockrate
    را به طور مختصر توضیح دهید و سپس بیان کنید که هر یک از گزینه‌های زیر چگونه بر این موارد تاثیر می‌گذارند.
    \begin{enumerate}
        \item   برنامه‌ی نوشته شده
        \item  سخت‌افزار ماشین
        \item کامپایلر
        \item مجموعه دستور‌العمل \lr{(ISA)}
    \end{enumerate}

\section[\lr{Performance}]{\lr{Performance}\RTLfootnote{\mahdi}}
    سه پردازنده‌ی مختلف به نام‌های \lr{P1}، \lr{P2} و \lr{P3} را در نظر بگیرید که یک 
    \lr{ISA}
    یکسان را با 
    \clockrate
    و
    \lr{CPI}های
    زیر اجرا می‌کنند.
\begin{table}[H]
    \begin{latin}
        \begin{center}
            \begin{tabular}{|c|c|c|}
                \hline
                Processor & Clock Rate & CPI \\
                \hline
                \hline
                P1  & 3 GHz & 1.5 \\
                \hline
                P2  & 2.5 GHz & 1.0 \\
                \hline
                P3  & 4 GHz & 2.2 \\
                \hline
            \end{tabular}
        \end{center}
    \end{latin}
\end{table}    

\begin{enumerate}
    \item 
    کدام پردازنده بالاترین \lr{performace} با یکای \lr{instructions per second} داراست؟
    \item 
    اگر هر کدام از این پردازنده‌ها به مدت ۱۰ ثانیه برنامه‌ای را اجرا کند، برای هر یک تعداد \lr{cycle}ها و تعداد دستوراتی که اجرا کرده‌اند \lr{(Instruction Count)} را بیابید.
    \item 
    فرض کنید در حال تلاش برای کاهش۳۰ درصدی زمان هستیم، اما با این کار \cpi اندازه‌ی \% ۲۰ درصد افزایش می‌یابد. برای اینکه بتوان موفق به به چنین کاهشی شد، \clockrate باید چقدر باشد؟
\end{enumerate}

\section[\lr{ISA}]{\lr{ISA}\RTLfootnote{\mahdi}}
دو پیاده‌سازی کاملا متفاوت از یک \isa را در نظر داشته باشید. در این \lr{ISA}، ۴ نوع کلاس دستوری متفاوت وجود دارد که در جدول زیر، برای هر یک از پیاده‌سازی‌ها \clockrate و \cpi آنها نوشته شده است.

\begin{table}[H]
    \begin{latin}
        \begin{center}
            \begin{tabular}{|c|c|c|c|c|c|}
                \hline
                Implementation & Clock Rate & CPI A & CPI B & CPI C & CPI D \\
                \hline
                P1 & 2.5 GHz & 1 & 2 & 3 & 3 \\
                \hline
                P2 & 3GHz & 2 &  2 & 2 &  2 \\
                \hline
            \end{tabular}
        \end{center}
    \end{latin}
\end{table}

\begin{enumerate}
    \item 
    اگر برنامه‌ای با تعداد 
    $10^6$
    دستور، که از 
    10\% دستورات کلاس \lr{A}،
    20\% دستورات کلاس \lr{‌B}،
    50\% دستورات کلاس \lr{‌C} و
    20\% دستورات کلاس \lr{D} 
    تشکیل شده باشد، کدام پیاده‌سازی سریع‌تر است؟
    \item 
    مقدر \cpi برای هر یک از پیاده‌سازی‌ها چقدر است؟
    \item 
    تعداد \lr{Clock Cycle} برای هر دو را پیدا‌ کنید.
\end{enumerate}

\section[سوال ترکیبی]{سوال ترکیبی\RTLfootnote{\hosna}}
جدول زیر اطلاعات مربوط به دو پردازنده را نشان می‌دهد.
\begin{table}[H]
    \begin{latin}
        \begin{center}
            \begin{tabular}{|c|c|c|c|}
                \hline
                Name & Execution Time (seconds) & CPI & Clock Rate \\
                \hline
                \hline
                A & 820 & 0.96 & 3 GHz \\
                \hline
                B & 580 & 2.94 & 3 GHz \\
                \hline
            \end{tabular}
        \end{center}
    \end{latin}
\end{table}

\begin{enumerate}
    \item 
         اگر با نرخ سیگنال (\lr{Clock Rate})
         ۴ گیگاهرتز،
          زمان اجرا ۱۰\% کاهش یابد (بدون آنکه مقدار \cpi تغییری کند،) تعداد دستورالعمل‌های هر برنامه را محاسبه کنید.
    \item 
    محاسبه کنید که اگر بخواهیم زمان \lr{CPU} را برای هر برنامه ۱۰\% کاهش دهیم (بدون آنکه تعداد دستورالعمل‌های آنها یا \cpi آنها تغییر کند،) نرخ سیگنال (\lr{Clock Rate}) چقدر باید باشد؟
    \item 
    محاسبه کنید که اگر مقدار \cpi به اندازه‌ی ۱۵\% و زمان \lr{CPU} به اندازه‌ی ۲۰\% کاهش یابد در حالی که تعداد دستورالعمل‌های هر برنامه ثابت بماند، نرخ سیگنال چقدر است؟
\end{enumerate}

\section[\lr{Benchmark}]{\lr{Benchmark}\RTLfootnote{\mahdi}}
جدول زیر نتایج 
\lr{SPEC CPU2006} 
را برای پردازنده‌ی 
\lr{AMD Barcelona}
نشان می‌دهد.

\begin{table}[H]
    \begin{latin}
        \begin{center}
            \begin{tabular}{|c|c|c|c|}
                \hline
                & program & Instrucion Count$\times 10^9$ & Execution Time (seconds) \\
                \hline
                \hline
                a. & bzip2 & 2389 & 750 \\
                \hline
                b. & go & 1658 & 700 \\
                \hline
            \end{tabular}
        \end{center}
    \end{latin}
\end{table}
    مقدار \cpi را پیدا کنید اگر نرخ سیگنال ۳ گیگاهرتز باشد.

جدول زیر، تعدادی بنچ‌مارک دیگر را نشان می‌دهد.

\begin{table}[H]
    \begin{latin}
        \begin{center}
            \begin{tabular}{|c|c|c|c|}
                \hline
                & program & CPI & Clock Rate \\
                \hline
                \hline
                a. & libquantum & 1.61 & 4 GHz \\
                \hline
                b. & astar & 1.79 & 4 GHz \\
                \hline
            \end{tabular}
        \end{center}
    \end{latin}
\end{table}

\begin{enumerate}
    \item 
    اگر تعداد دستورات هر بنچ‌مارک 10\% افزایش یابد (بدون اینکه \cpi تغییری کند،) مقدار \lr{CPU time} چقدر افزایش می‌یابد؟
    \item
      اگر تعداد دستورات هر بنچ‌مارک 10\% و \cpi هم ۵\% افزایش یابد، مقدار \lr{CPU time} چقدر افزایش می‌یابد؟
    
\end{enumerate}
\end{document}